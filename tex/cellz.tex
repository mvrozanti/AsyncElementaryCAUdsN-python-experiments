\documentclass[a4paper]{ltxdoc}
\usepackage{lmodern}			% Usa a fonte Latin Modern			
\usepackage[T1]{fontenc}		% seleção de códigos de fonte.
\usepackage[utf8]{inputenc}		% determina a codificação utiizada (conversão automática dos acentos)
\usepackage{hyperref}  			% controla a formação do índice
\usepackage{parskip}			% espaçamento entre os parágrafos
\usepackage{microtype} 			% para melhorias de justificação
\usepackage{morefloats}			% permite mais floats
\usepackage[absolute]{textpos}
\usepackage{tabularx}
\usepackage[table]{xcolor}
\usepackage{graphicx}
\usepackage{mathtools}
\usepackage{multirow}

% Babel e ajustes
\usepackage[english,brazil]{babel}		% idiomas
\addto\captionsbrazil{
    %% ajusta nomes padroes do babel
    \renewcommand{\bibname}{Refer\^encias}
    \renewcommand{\indexname}{\'Indice}
    \renewcommand{\listfigurename}{Lista de ilustra\c{c}\~{o}es}
    \renewcommand{\listtablename}{Lista de tabelas}
    %% ajusta nomes usados com a macro \autoref
    \renewcommand{\pageautorefname}{p\'agina}
    \renewcommand{\sectionautorefname}{se{\c c}\~ao}
    \renewcommand{\subsectionautorefname}{subse{\c c}\~ao}
    \renewcommand{\paragraphautorefname}{par\'agrafo}
    \renewcommand{\subsubsectionautorefname}{subse{\c c}\~ao}
    \renewcommand{\paragraphautorefname}{subse{\c c}\~ao}
}  

\usepackage{color}
\definecolor{thered}{rgb}{0.65,0.04,0.07}
\definecolor{thegreen}{rgb}{0.06,0.44,0.08}
\definecolor{thegrey}{gray}{0.5}
\definecolor{theshade}{rgb}{1,1,0.97}
\definecolor{theframe}{gray}{0.6}
\definecolor{blue}{RGB}{41,5,195}
% \newcommand{\orientador}{Prof. Dr. Calebe de Paula Bianchini}
\newcommand{\orientador}{Prof. Dr. Pedro Paulo Balbi de Oliveira}
\newcommand\tab[1][1cm]{\hspace*{#1}}
\IfFileExists{listings.sty}{
  \usepackage{listings}
\lstset{%
	language=[LaTeX]TeX,
	columns=flexible,
	basicstyle=\ttfamily\small,
	backgroundcolor=\color{theshade},
	frame=single,
	tabsize=2,
	rulecolor=\color{theframe},
	title=\lstname,
	escapeinside={\%*}{*)},
	breaklines=true,
	commentstyle=\color{thegrey},
	keywords=[0]{\fichacatalografica,\errata,\folhadeaprovacao,\dedicatoria,\agradecimentos,\epigrafe,\resumo,\siglas,\simbolos,\citacao,\alineas,\subalineas,\incisos},
	keywordstyle=[0]\color{thered},
	keywords=[1]{},
	keywordstyle=[1]\color{thegreen},
	breakatwhitespace=true,
	alsoother={0123456789_},
	inputencoding=utf8,
	extendedchars=true,
	literate={á}{{\'a}}1 {ã}{{\~a}}1 {é}{{\'e}}1 {è}{{\`{e}}}1 {ê}{{\^{e}}}1 {ë}{{\¨{e}}}1 {É}{{\'{E}}}1 {Ê}{{\^{E}}}1 {û}{{\^{u}}}1 {ú}{{\'{u}}}1 {â}{{\^{a}}}1 {à}{{\`{a}}}1 {á}{{\'{a}}}1 {ã}{{\~{a}}}1 {Á}{{\'{A}}}1 {Â}{{\^{A}}}1 {Ã}{{\~{A}}}1 {ç}{{\c{c}}}1 {Ç}{{\c{C}}}1 {õ}{{\~{o}}}1 {ó}{{\'{o}}}1 {ô}{{\^{o}}}1 {Õ}{{\~{O}}}1 {Ó}{{\'{O}}}1 {Ô}{{\^{O}}}1 {î}{{\^{i}}}1 {Î}{{\^{I}}}1 {í}{{\'{i}}}1 {Í}{{\~{Í}}}1,
}
\let\verbatim\relax
 	\lstnewenvironment{verbatim}[1][]{\lstset{##1}}{}
}

\usepackage[alf]{abntex2cite}	% citacoes


% COMANDOS PROPRIOS
% \newcommand{\abnTeX}{}
% \newcommand{\abnTeXForum}{}
\newcommand{\abnTeXSite}{}

\title{\protect\parbox{\textwidth}{\protect\centering \textbf{Conservabilidade de estados de autômatos celulares elementares com atualizações assíncronas por prioridade da vizinhança}}}

\author{Marcelo Vironda Rozanti\\\abnTeXSite Felipe Stefanelli de Aguiar Silva\\\abnTeXSite} 

\date{\today}

\hypersetup{
  pdftitle={Conservabilidade de estados de autômatos celulares elementares com atualizações assíncronas por prioridade da vizinhança},
  pdfauthor={Marcelo Vironda Rozanti}{Felipe Stefanelli de Aguiar Silva},
  pdfkeywords={Asynchronous priority based updating, Elementary Cellular Automata, New Kind of Science, Discrete dynamical systems, Number-conserving}, 
  pdfproducer={Marcelo Vironda Rozanti},
  pdfcreator={vim+LaTeX+abnTeX2},
  colorlinks=true,
  linkcolor=black,
  citecolor=black,
  urlcolor=blue
}

% \EnableCrossrefs
\CodelineIndex
% \RecordChanges
\hyphenpenalty=1000

\begin{document}

\thispagestyle{empty}
\huge{\centerline{UNIVERSIDADE PRESBITERIANA MACKENZIE}}

\bigskip
\bigskip

\centerline{\large{FACULDADE DE COMPUTAÇÃO E INFORMÁTICA}}

\makeatletter
\vfill
\large{\centering{\@author}}

\vspace*{\fill}

\large{Conservabilidade de estados de autômatos celulares elementares com}
\large{\centerline{atualizações assíncronas por prioridade da vizinhança}}

\vspace*{\fill}
\large{\centerline{SÃO PAULO}}
\large{\centerline{2019}}

\pagebreak

\large{\centering{\@author}}
\vfill
\large{\centerline{\@title}}
\vfill
\vspace*{\fill}

Orientador: \orientador
\\
\\
\\


\large{\centerline{SÃO PAULO}}
\large{\centerline{2019}}

\begin{textblock}{6}(9,9)
% Projeto do Trabalho de Conclusão de Curso apresentado à Faculdade de Computação e Informática da Universidade Presbiteriana, desenvolvida como requisito parcial para aprovação na disciplina de Metodologia da Pesquisa em Computação, ministrada pelo professor Dr. Everton Knihs, do curso de Ciência da Computação.
\end{textblock}

\maketitle

\begin{abstract}
Autômatos Celulares são sistemas computacionais discretos que se têm provado úteis como modelos genéricos de complexidade e representação de diversas dinâmicas em uma varidade de áreas científicas. Estes sistemas podem ser especificados puramente em termos matemáticos e até implementados em estruturas físicas. Muitos deles podem computar funções e resolver problemas algorítmicos. O presente projeto explora um conjunto fundamental deles, chamados Automatos Celulares Elementares com um tipo específico de atualização assíncrona baseada em prioridade da vizinhança com a esperança de encontrar modelos conservativos que podem ser usados em aplicações práticas relativas a . 
% O código correspondende a este trabalho pode ser acessado no repositório \url{https://github.com/mvrozanti/TCC}.
\end{abstract}
Palavras-chave: Autômatos celulares elementares, atualização assíncrona por prioridade da vizinhança, New Kind of Science, Sistemas dinâmicos discretos, Conservabilidade

\vfill

\selectlanguage{english}
\begin{abstract}
Cellular Automata are discrete computational systems that have proved useful as general models of complexity and representations of dynamics on a variety of scientific fields. These systems can be specified in purely mathematical terms and be implemented in physical structures. Many of them can compute functions and solve algorithmic problems. The present project attempts to explore a fundamental subset of them, called Elementary Cellular Automata with a specific kind of neighbourhood-priority-based asynchronous updating in the search of number-conserving models, which can be used for a variety of practical applications. 
%The implementation developed is hosted at \url{https://github.com/mvrozanti/TCC}.
\end{abstract}
Keywords: \@pdfkeywords

\break

\selectlanguage{brazil}
\renewcommand{\contentsname}{\centerline{\Large Sumário}}

\tableofcontents

\listoftables

\listoffigures

\break

\section{INTRODUÇÃO}

% \tab Este projeto teve como objetivo explorar propriedades conservativas em autômatos celulares (ACs) elementares com atualização assíncrona baseada em prioridade da vizinhança. 

\tab Autômatos Celulares (ACs) são uma categoria de sistemas discretos. Pela sua própria simplicidade, esses sistemas têm ocupado uma posição privilegiada no estudo de complexidade nos mais diversos campos da ciência, de biologia teórica a economia entre muitos outros. O desenvolvimento de ACs é tipicamente atribuído a John von Neumann através de suas tentativas de desenvolver um modelo abstrato de autoreprodução biológica. Ao fim de 1950, foi notado que ACs poderiam ser vistos como computadores paralelos \cite[p. 876]{wolfram2002new}. Apesar de falta de investigação científica até 1970, um exemplo de AC se tornou muito famoso por seu comportamento complexo e regras simplesmente descritas inventadas por John Conway, \textit{The Game of Life}, que se popularizou após sua aparição na revista \textit{Scientific American}. 

\tab Desde então, múltiplos estudos compreensivos foram realizados por cientistas ao redor do mundo, destacando os trabalhos feitos por Stephen Wolfram na década de 1980, culminando na publicação do livro A New Kind of Science em que Wolfram apresenta uma coleção de resultados a respeito de ACs, com uma série de descobertas revolucionárias. 

% \tab No capítulo \ref{referencial}, é apresentado alguns conceitos básicos referentes à definição dos ACs. 

\subsection{CONTEXTUALIZAÇÃO E RELEVÂNCIA} \label{contex}

\tab A motivação inicial para o estudo de ACs por parte de von Neumann não era matemática mas tinha como diretriz encontrar uma forma viável de tratar do problema de como fazer máquinas se reproduzirem. Hoje é sabido que esses sistemas são aplicáveis para resolver problemas relativos a classificação de densidade, simulação de partículas, compressão de dados, design estrutural, representação de comportamentos dinâmicos como trânsito em cidades ou movimentação de seres vivos entre muitos outros.

\tab Há uma infinitude de ACs possíveis de serem engenhados. Um problema de possível realização através de ACs com propriedades conservativas, como é notado por \citeonline{yuen2009applications}, é na modelagem de trânsito de veículos: ``One approach to reducing congestion would be to construct new or widen existing roads to provide additional lanes and hence increase the capacity of the road infrastructure. However, this approach can be very costly and delays may  worsen  while  road  works  are  in  progress.   Sometimes  it  is  difficult  to improve the existing road system due to environmental or social objectives.  Another approach may be to control the traffic in such a way that congestion would be solved with the use of traffic lights or making adjustments to road marking.  However, it is not a simple task to decide which approach would be most effective for a particular road network in order to limit traffic congestion.  The use of a traffic model is needed to predict the behaviour of vehicles and the interactions between them on the roads.''

\break

\subsection{OBJETO DE PESQUISA} \label{objetopesq}

\tab O objeto de pesquisa deste trabalho são ACEs com atualização assíncrona por prioridade da vizinhança.

\subsubsection{PROBLEMA DE PESQUISA} \label{problema}
Dentre o conjunto de esquemas de atualização não-equivalentes, encontrar quais pares \texttt{(R,E)} são conservativos.    

\break

% \subsubsection{FÓRMULAS UTILIZADAS} \label{formulas}

% \break

\subsubsection{HIPÓTESE BÁSICA} 

\tab Dados os esquemas de atualização mencionados na seção \ref{objetopesq}, encontrar quais esquemas apresentaram conservabilidade numérica <<isso nao eh uma hipotese>>.

\subsubsection{VARIÁVEIS} \label{varia}
A implementação do simulador tem como parâmetros os seguintes itens:

\begin{itemize}
	\item Largura do espaço celular\\
          \tab N, como definido em \ref{referencial}.
	\item Espaço Celular:\\
          \tab 
	\item Regra de transição:\\
          \tab 
	\item Timesteps:\\
          \tab
\end{itemize}

\subsection{OBJETIVOS DO ESTUDO} 
O objetivo do presente trabalho foi identificar e quantificar 

\subsubsection{OBJETIVO GERAL}
<+>

\subsubsection{OBJETIVOS ESPECÍFICOS}
<+>

\tab Para alcançar o objetivo geral proposto para a resolução do problema de pesquisa, os seguintes objetivos específicos foram estabelecidos:

\begin{itemize}
    \item Definição formal da estratégia de atualização.
    \item Definição de conservabilidade.
    \item Identificação de conservabilidade.
    \item Renderização das evoluções temporais.
    \item Otimização da pesquisa por via de filtragem de esquemas equivalentes.
\end{itemize}

\subsection{JUSTIFICATIVA}
<+>

\subsection{DELIMITAÇÃO DO ESTUDO} \label{delimitacao}
<+>
% \tab Deve-se restringir a implementação dos algoritmos evidenciados no item \ref{problema}:

% \begin{itemize}
% \end{itemize}

% \break

\subsection{ORGANIZAÇÃO DO ESTUDO}

Descrição dos capítulos:

\begin{enumerate}
    \item Introdução
        \subitem Preâmbulo deste trabalho.
    \item Referencial Teórico
        \subitem Sustentação argumentativa sobre o tema proposto.
    \item Metodologia de Pesquisa
        \subitem Sistematização dos instrumentos e processos de estudo empregados no presente trabalho.
    \item Cronograma
        \subitem Planejamento das tarefas necessárias para a conclusão deste trabalho, bem como suas expectativas de início e conclusão.
\end{enumerate}
 
\section{REFERENCIAL TEÓRICO} \label{referencial}
\tab Em relação ao referencial teórico estudado neste TCC, foi de suma importância entender os seguintes tópicos:

\subsection{AUTÔMATOS CELULARES} \label{ref_ac}

Um AC pode ser descrito simplesmente por seu \textit{espaço celular} e \textit{regra de transição}. O espaço celular é um conjunto finito \textit{d-dimensional} ordenado de células onde cada célula no índice \textbf{i} tem um estado entre \textit{k} estados. As regras de transição definem o novo estado de uma determinada célula em função de sua vizinhança de tamanho \textit{r}, dito o raio do AC. O comportamento da aplicação das regras de transição nas bordas do espaço pode ser especificado de várias formas, muitas vezes como conexo com a borda contrária tornando o espaço efetivamente contínuo.

% \cite{wolfram2002new}

% \tab ``O \textit{espaço celular} é um reticulado de N células idênticas, cada qual com um padrão idêntico de conexões locais para outras células e com condições de contorno. O conjunto de estados possíveis da célula é denotado por \(\Sigma\) e o número de estados desse conjunto é denotado por \textbf{k}. Cada célula é denotada por um índice \textbf{i} e seu estado a um dado tempo \textbf{t} é denotado por \(n_i^t\), onde \(S_i^t \in \Sigma\). O estado \(S_i^t\) da célula \textbf{i} e os estados das células às quais a célula \textbf{i} está conectada são chamados de vizinhança \(n_i^t\) da célula \textbf{i}.
%
% \tab A \textit{regra de transição} é denotada por \(\Phi(n_i^t)\) que fornece o próximo estado \(S_i^{t+1}\) para cada célula \textbf{i}, como uma função de \(n_i^t\). A cada passo de tempo, todas as células atualizam seus estados sincronicamente de acordo com \(\Phi(n_i^t)\). 

% \tab Para ACs unidimensionais, o tamanho da vizinhança \(n_i\) é normalmente escrito como \(\lvert n_i\rvert = 2r + 1\), onde \textbf{r} é chamado o raio do AC.''


% (como é descrito\ref{varia}).

\subsubsection{AUTÔMATOS CELULARES ELEMENTARES}
ACEs formam a família de ACs com \textit{d}=1, \textit{k}=2 e \textit{r}=1, ou seja, são unidimensionais, com células binárias e vizinhanças são definidas por 3 células contíguas.

\tab Visualização:

\begin{figure}[!htbp]
\centerline{\includegraphics[width=\linewidth/2]{imgs/256ecas.png}}
\caption{Ilustração das 256 regras elementares}
\end{figure}

\subsubsection{REPRESENTAÇÕES}

<+>

Então, para ACs de 1 dimensão, elementares, com vizinhança \texttt{\{-1,0,1\}}, com conjunto de estados \texttt{\{0,1\}}, na ordem lexicográfica \{\textit{\(w_0\)}, ..., \textit{\(w_7\)},\} em \(\{0,1\}^3\), se \(\delta(\textit{w}_i)=s_i,s_0,...s_7\) representa \(\delta\). Analogamente, qualquer inteiro positivo menor que 256 = \(2^8\) define um ACE \cite[cap. 2]{delorme1999introduction}.

\begin{figure}[!htbp]
\centerline{\includegraphics[width=\linewidth/2]{imgs/typical-config.png}}
\caption{Nas renderizações, é convenção representar 1s como células em preto e 0s em branco. \cite{wolfram2002new} }
\end{figure}

\begin{figure}[!htbp]
\centerline{\includegraphics[width=\linewidth/2]{imgs/rule_22.png}}
\caption{Ilustração do mapa de transições para a regra 22 na notação de \citeonline{wolfram1983cellular}}
\end{figure}

\begin{figure}[!htbp]
\centerline{\includegraphics[width=\linewidth/2]{imgs/rule-transitions.png}}
\caption{Ilustração do mapa de transições para a regra 90 na notação de \citeonline{wolfram1983cellular}}
\end{figure}

\subsubsection{TIPOS DE ATUALIZAÇÃO} \label{tiposatt}
\tab Parte fundamental da \textit{regra de transição} de um AC é a ordem de atualização das células no espaço celular. Apesar de serem fundamentalmente concebidos com o comportamento de atualização síncrona, é possível se utilizar das mais diversas e criativas estratégias de atualização.

\tab  No presente trabalho foi empregado um tipo inédito de estratégia de atualização: em função da prioridade da vizinhança, que consiste em cada regra de transição estar associada a uma prioridade no intervalo \(\left[1, 8\right]\). Esse tipo de atualização requer uma distinção dos timesteps globais e locais (microtimesteps):  as prioridades de vizinhanças são iteradas sequencialmente de menor a maior e para cada prioridade, o espaço celular é varrido e atualizam-se apenas as células que tenham vizinhança com a prioridade atual e mantém-se o estado das que não vizinhanças que não estejam na prioridade atual da iteração. Ao fim da varredura do espaço celular, o processo se repete para a próxima prioridade, agora iterando sobre o reticulado gerado no processo descrito anteriormente. O reticulado resultante após iterar sobre todas as prioridades é chamado de timestep global (ou macrotimestep) e é incluído nas renderizações, o que não acontece com os microtimesteps.

\break 

\subsection{CONSERVABILIDADE NUMÉRICA} \label{num-conserv}
Segundo \citeonline{boccara2002number}, ``A one-dimensional q-state n-input CA rule f is number-conserving if, for all cyclic configurations of length L \(\geq\) n, it satisfies'':

\[f(x_1,x_2,...,x_{n-1},x_n) + f(x_2,x_3,...,x_n,x_{n+1}) + ... \]
\tab \[ + \;   f(x_L,x_1,...,x_{n-2},x_{n-1} = x_1 + x_2 + ... + x_L\]

Como é mencionado no item \ref{contex}, ACs conservativos podem ser usados para modelar ambientes em que há conservabilidade, como trânsito automobilístico, simulação de partículas, entre outros.

\section{METODOLOGIA DA PESQUISA}
\tab No que tange à Metodologia empregada neste TCC, o trabalho teve início com uma revisão da literatura específica sobre o tema da pesquisa. Esta pesquisa abrange conceitos fundamentais de teoria da computação e o ``estado da arte'' em termos de análise de conservabilidade.

Este alicerce teórico foi obtido através de autores como: Wolfram; Oliveira;
entre outros, além de pesquisas em periódicos científicos, sites, publicações em empresas, teses e dissertações em universidades e publicações de associações técnicas.  A leitura, análise e comparação da fundamentação teórica tiveram início com ACEs síncronos, modificados em seguida para viabilizar o tipo de atualização mencionado no item \ref{tiposatt}.


\subsection{ETAPAS DA PESQUISA}
\tab Para definir as etapas da pesquisa, foi necessário atender às delimitações de estudo (item \ref{delimitacao}), desenvolvendo uma implementação que atendesse à execução dos ACE com esquemas distintos <+>, como descrito no Cronograma apresentado no item \ref{cron}.

\break

Assim, pode-se dizer que as etapas desenvolvidas neste estudo foram:

\begin{enumerate}
    \item Estudo do estado da arte;
    \item Implementação do simulador para execuções síncronas;
    \item Implementação do renderizador;
    \item Implementação do simulador para execuções assíncronas;
    \item Implementação do verificador de evoluções conservativas;
    \item Análise e testes das implementações;
    \item Análise dos resultados;
    \item Artigo.
\end{enumerate}
%
% \subsubsection{CONCEITOS EMPREGADOS}
% <+>
%
% \begin{itemize}
%   \item 
% \end{itemize}

\subsection{CLASSIFICAÇÃO DA PESQUISA}
\tab O tempo total previsto para a conclusão desta pesquisa é de 1 ano, como mostrado no capítulo \ref{cron}.

\subsubsection{NATUREZA DA PESQUISA}
\tab Esta é uma pesquisa exploratória, dada a natureza desconhecida das propriedades de ACEs com esse tipo de atualização.

\subsubsection{ABORDAGEM} 
\tab Esta pesquisa é baseada em cálculos, medidas objetivas e dados verificáveis.

\subsubsection{FINS} 
\tab Esta pesquisa foi voltada para encontrar caminhos, formas, maneiras e procedimentos para atingir um determinado fim, buscando definir um processo ou uma ferramenta que leve à solução do problema proposto (\ref{problema}).

\subsubsection{MEIOS} 
\tab Quanto aos meios, foram utilizados os recursos mencionados na Bibliografia.

\subsubsection{PESQUISA BIBLIOGRÁFICA}
<+>

% \vfill
\break
\section{CRONOGRAMA} \label{cron}

\tab As atividades desta pesquisa se desenvolveram de acordo com o cronograma apresentado a seguir, no prazo de 12 meses:

\begin{table}[h]
    \caption{Cronograma de atividades}
    \centerline{
        \begin{tabular}{|l|l|l|l|l|l|l|l|l|l|l|l|l|}
          \hline
          \multirow{2}{*}{ATIVIDADE}                              & \multicolumn{12}{|c|}{MÊS} \\
          % \hline
                                                                  & 1                             & 2                  & 3                  & 4                  & 5                  & 6                  & 7                  & 8                  & 9                  & 10                 & 11                 & 12 \\
          \hline
          Estudo do estado da arte                                & \cellcolor{yellow}            & \cellcolor{yellow} &                    &                    &                    &                    &                    & \cellcolor{yellow} &                    &                    &                    & \\
          \hline
          Implementação do simulador para execuções síncronas     &                               & \cellcolor{yellow} & \cellcolor{yellow} &                    &                    &                    &                    &                    &                    &                    &                    & \\
          \hline
          Implementação do renderizador                           &                               & \cellcolor{yellow} & \cellcolor{yellow} &                    &                    &                    &                    &                    &                    &                    &                    & \\
          \hline
          Implementação do simulador para execuções assíncronas   &                               &                    &                    & \cellcolor{yellow} & \cellcolor{yellow} &                    &                    &                    &                    &                    &                    & \\
          \hline
          Implementação da verificador de evoluções conservativas &                               &                    &                    & \cellcolor{yellow} & \cellcolor{yellow} &                    &                    &                    &                    &                    &                    & \\
          \hline
          Análise e testes das implementações                     &                               &                    &                    &                    &                    & \cellcolor{yellow} & \cellcolor{yellow} &                    &                    &                    &                    & \\
          \hline
          Análise dos resultados                                  &                               &                    &                    &                    &                    & \cellcolor{yellow} & \cellcolor{yellow} & \cellcolor{yellow} & \cellcolor{yellow} & \cellcolor{yellow} &                    & \\
          \hline
          Artigo                                                  &                               &                    &                    &                    &                    &                    &                    & \cellcolor{yellow} & \cellcolor{yellow} & \cellcolor{yellow} & \cellcolor{yellow} & \cellcolor{yellow}\\
          \hline
    \end{tabular}
  }
\end{table}

% \break

\addcontentsline{toc}{section}{Referências}
\bibliography{cellz} \label{bib}

\end{document}
